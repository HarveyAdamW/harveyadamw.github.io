% Awesome Source CV LaTeX Template
%
% This template has been downloaded from:
% https://github.com/darwiin/awesome-neue-latex-cv
%
% Author:
% Christophe Roger
%
% Template license:
% CC BY-SA 4.0 (https://creativecommons.org/licenses/by-sa/4.0/)

%Section: Work Experience at the top
\section{\texorpdfstring{\color{Blue}Research Experience}{Research Experience}}
\begin{longtable}{R{2.5cm}|p{14.8cm}}
 	\textsc{2017-Present}& \emph{Graduate Research Assistant} UMBC\\&
 	\footnotesize{
 		\textbf{Proper motions of optical-UV AGN jets using Hubble Space Telescope data.}
 		\begin{itemize}
			\item Registered images of AGN jets using background globular clusters to achieve an astrometric accuracy of ~0.1 pixels.
            \item Developed a utility to locate globular clusters in an image to a high precision with filters for which a well-modeled point-spread function was either not available or the photon counts too low to fit with a detailed point-spread function.
            \item Debugged legacy code to fix errors in photon counts in stacked images.
            \item Modeled bright host galaxy light for subtraction to be able to measure the relatively faint jet.
		\end{itemize}
		\vspace{-1em}
        }\\&
        \footnotesize{
     		\textbf{Constraining the dominant location of kinetic energy dissipation in powerful blazars.}
     		\begin{itemize}
    			\item Used a diagnostic (the seed factor) dependent on only observables of a blazar SED to constrain the dominant location of energy dissipation.
    			\item Fit blazar SEDs with empirical models using maximum likelihood estimation with a simulated annealing algorithm.
    			\item Developed a method of error estimation for the peak frequencies and peak fluxes of these SEDs using Wilk's theorem, the profile-likelihood method, and a modified non-parametric bootstrapping.
    			\item Developed a kernel density estimation technique using a modified non-parametric bootstrapping technique.
                \item Used bootstrapping to test the median value of the observed distribution of the seed factor against the expected values for the broad-line region and the molecular torus, finding that the broad-line region is significantly rejected, and the the molecular torus is compatible with the distribution median.
    		\end{itemize}
    		\vspace{-1em}

	}\\
 \multicolumn{2}{c}{} \\
\end{longtable}
